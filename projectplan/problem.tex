

\section{Problem}
% Det här avsnittet är ofta den viktigaste delen av planeringsrapporten (och av den slutgiltiga uppsatsen/rapporten). Den syftar till att identifiera frågan/frågorna som ska tas upp i projektet. Det är viktigt att gruppen gör en problemanalys även om det i projektförslaget redan finns ett problem (en uppgift) specificerat. Anledningen till detta är att det riktiga primära problemet ofta skiljer sig från det i början av uppdragsgivaren/förslagsställaren/kunden föreslagna. Problemanalysen syftar också till att bryta ner problemet/uppgiften i mindre och mer detaljerade delproblem/deluppgifter, vilket också leder till formulering av delsyften. Genom att göra detta får studenterna mycket bättre förståelse för de olika aspekterna av problemet/uppgiften. Utan den här informationen är det omöjligt att identifiera vilken information som behövs, vilka informationskällor som behövs och lämpliga tillvägagångssätt.

% En bra problemanalys som identifierar delproblem/deluppgifter och delsyften vilar i många fall på användning av teorier och modeller från litteraturen. En litteraturgenomgång bör därför genomföras tidigt i processen.

\subsection{Conceptual Complexity}
In classical physics, if we know the state of a system at time t=0, we can predict its state at any later time, like t=1, with certainty. Quantum physics, however, introduces non-determinism: the same initial conditions can lead to different outcomes, each with a certain probability.

The probabilistic nature of the quantum realm might confuse the reader and make the concepts hard to understand at first.

An example of the probabilistic nature of quantum mechanics is when we measure a qubit that is in a superposition. In quantum computing, a qubits state is described as a superposition of $\ket{0}$ and $\ket{1}$ \cite{niel_chang}.

$$ \ket{\psi} = \alpha  \ket{0} + \beta \ket{1}$$

When we perform a measurement on such a qubit, the probability of it collapsing into the state $\ket{0}$ is given by $|\alpha|^2$, while the probability of it collapsing into the state $\ket{1}$ is $|\beta|^2$. This means that the outcome of the measuring can't be known before.

The concept of teleportation or instantaneous transfer of information often falls into the realm of fantasy or science fiction. However, in quantum mechanics, the phenomenon known as quantum teleportation is a well-established technique for transmitting the information of a quantum state across a distance, as documented in various studies like \cite{quanttelep}. It's important to understand, however, that this doesn't make faster then light communication possible. The no-signaling theorem ensures that information transfer faster than light remains impossible. Therefore, despite its intriguing name, quantum teleportation doesn't represent actual teleportation but rather a fascinating quantum phenomenon.

These conflicting descriptions of our reality increase the conceptual complexity of quantum mechanics and might make it a more intimidating subject to take on.



\subsection{Lack of Interactive Tools}
The current state of quantum computer simulators is still in an early state and lacks any tool accessible to beginners to the subject. This leads to a large hurdle for people wishing to learn more about quantum computers and may discourage people from pursuing this subject. Tools on the market usually implement a highly programmatic interface that might cater to people familiar to programming but often lack the visual component important for learning. Example of these are Googles qsim\cite{googleqsim} or Intel Quantum Simulator\cite{intelqsim}.

There are some visual based tools out right now like IBM's Quantum Composer \cite{ibmqsim} or Quirk, an open source drag and drop quantum simulator \cite{openqsim}. However they both lack any clear explanations of the functions of quantum gates or what all values on the screen mean and can be intimidating for beginners.
