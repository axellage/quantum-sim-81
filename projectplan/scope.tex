\section{Scope}
% Avgränsningarna ska ta upp vilka delar av problemet som inte tas upp i uppsatsen/rapporten, och anledningen till detta. Motivering av avgränsningarna är viktigt.
\begin{comment}
    Since the available computing power is very limited, a fair limitation is to aim at a simulator managing 10 qubits simultaneously. Det är både backend och frontend. Kan inte förenkla genom att multiplicera matriser i förväg eftersom vi vill ha med alla steg, inte kunna spara sina egna circuits, men inbyggda circuits typ shor's osv. svårt att visa statevector, bara mäter i absoluta slutet i början av applikationen, kan inte visa 10 qubits i grafer max 6 st verkar rimligt, samt timestep endast till 6 qubits då det blir för svårt att visa grafiskt med fler qubits, programmet är tänkt till gymnasie, högskola/universitets-nivå för fritids/projektsbruk
\end{comment}
Since the available computing power is limited, a fair limitation is to aim for a simulator managing 10 qubits simultaneously. Since the amount of memory used, and therefore time taken to simulate, grows exponentially when adding qubits it quickly scales the computers used. In addition to this will 10 qubits require $2^{10}$ binary combinations to be calculated but also displayed. Therefore is another limitation that the simulator will cap at displaying the results graphically at 6 qubits which is 64 states. 

% det här stycket kan vara lite svårt att förstå om man inte är insatt
A feature that is aimed towards implemented is the ability to view the circuit and the qubits' state during certain timesteps along the circuit, in order to understand how the qubits are manipulated along the way. This does set a constraint on the backend since it needs to calculate every matrix operation as it is encountered along the circuit. Otherwise one could calculate certain matrix operations beforehand in order to optimise runtime and memory usage. In addition to this the main feature of the simulator will be one that a user can drag and drop and build its own circuit, but one limitation is that a user will not be able to save its circuit for future use. The user will instead have to build it again. However, there will be pre-programmed circuits available in the simulator to showcase some scientifically important results in quantum computing, such as Shor's algorithm.